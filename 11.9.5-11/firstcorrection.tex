\documentclass{article}
\usepackage{amsmath}
\usepackage{array}

\begin{document}

\title{Maths Assignment}
\author{Abhignya Gogula\\
        EE23BTECH11023}
\maketitle

\section*{Problem Statement}
A G.P consists of an even number of terms. If the sum of all terms is 5 times the sum of terms occupying odd places, then find its common ratio.

\section*{Solution}
Let \( x(0) \) denote the first term of the geometric progression and \( r \) the common ratio. The sum of a geometric progression with \( n \) terms can be calculated using the formula:
\[ S_n = x(0) \frac{{r^n - 1}}{{r - 1}} \]

The given equations are as follows:

\[
S_n = x(0) \frac{r^n - 1}{r - 1}, \quad S_{\text{odd}} = x(0) (1 + r^2 + r^4 + \ldots + r^{n-2})
\]

The condition is: $S_n = 5  S_{\text{odd}}$.

Now, we'll express these equations in terms of Z-transforms:

\[
X(z) = x(0) \frac{1 - r^nz^{-1}}{1 - rz^{-1}}, \quad X_{\text{odd}}(z) = x(0) (1 + r^2z^{-2} + r^4z^{-4} + \ldots + r^{n-2}z^{-(n-2)})
\]

To equate $S_n$ and $5  S_{\text{odd}}$ using Z-transforms, we manipulate the Z-transforms to find a relationship involving $z$ and the common ratio $r$:

\[
x(0) \frac{r^n - 1}{r - 1} = 5  x(0) (1 + r^2z^{-2} + r^4z^{-4} + \ldots + r^{n-2}z^{-(n-2)})
\]

Simplifying further:

\[
\frac{r^n - 1}{r - 1} = 5  (1 + r^2z^{-2} + r^4z^{-4} + \ldots + r^{n-2}z^{-(n-2)})
\]

By considering the geometric series in the parentheses:

\[
1 + r^2z^{-2} + r^4z^{-4} + \ldots + r^{n-2}z^{-(n-2)} = \frac{1 - r^n z^{-n}}{1 - r^2z^{-2}}
\]

Substituting this back:

\[
\frac{r^n - 1}{r - 1} = 5 \frac{1 - r^n z^{-n}}{1 - r^2z^{-2}}
\]

Expanding and rearranging:

\[
r^n - 5r + r^{n+2}z^{-2} - r^2z^{-2} +6 = 0
\]

Let's assume a sequence \( x(n) \) given by \( x(n) =x(0)^n u(n) \), where \( x(0) \) is a constant and \( u(n) \) is the unit step function.

The Z-transform is given by:
\[ X(z) = \sum_{n=0}^{\infty} x(n)z^{-n} = \sum_{n=0}^{\infty} (x(0)z^{-1})^n = \frac{1}{1 - x(0)z^{-1}} \]

This represents the Z-transform for the given sequence \( x(n) = x(0)^n u(n) \).

\section*{Desired Sum Using $X(z)$}
The Z-transform of the sequence \( x(n) = x(0)^n u(n) \) is given by:
\[ X(z) = \sum_{n=0}^{\infty} x(n)z^{-n} = \frac{1}{1 - az^{-1}} \]

To obtain the desired sum, let's perform the inverse Z-transform by expressing \( X(z) \) in partial fractions:
\[ X(z) = \frac{1}{1 - x(0)z^{-1}} = \frac{A}{1 - az^{-1}} \]

To find \(A\), multiply both sides by the denominator:
\[ 1 = A(1 - x(0)z^{-1}) \]
\[ A = 1 \]

Therefore, the partial sum using \(X(z)\) is \(x(n) = x(0)^n u(n)\).


\begin{table}[h!]
\centering
\begin{tabular}{|c|c|}
\hline
Parameter & Description \\
\hline
\( n \) & Number of terms in the G.P (positive even integer) \\
\hline
\(x(0) \) & first term in the G.P \\
\hline
\( r \) & common ratio in the G.P \\
\hline
\( x(n) \) & nth term in the G.P \\
\hline
\( X(z) \) & Z transform of X(n) \\
\hline
\end{tabular}
\end{table}

\end{document}



