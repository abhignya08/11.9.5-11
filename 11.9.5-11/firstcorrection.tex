\documentclass{article}
\usepackage{amsmath}
\usepackage{array}

\begin{document}

\title{Maths Assignment}
\author{Abhignya Gogula\\
        EE23BTECH11023}
\maketitle

\section*{Problem Statement}
A G.P consists of an even number of terms. If the sum of all terms is 5 times the sum of terms occupying odd places, then find its common ratio.

\section*{Solution}
\begin{table}[h!]
\centering
\begin{tabular}{|c|c|}
\hline
Parameter & Description \\
\hline
\( n \) & Number of terms in the G.P (positive even integer) \\
\hline
\(x(0) \) & first term in the G.P \\
\hline
\( r \) & common ratio in the G.P \\
\hline
\( x(n) \) & nth term in the G.P \\
\hline
\( X(z) \) & Z transform of X(n) \\
\hline
\end{tabular}
\end{table}

Let \( x(0) \) denote the first term of the geometric progression and \( r \) the common ratio. The sum of a geometric progression with \( n \) terms can be calculated using the formula:
\begin{align}
 S_n = x(0) \frac{{r^n - 1}}{{r - 1}} \\
X(z) &= x(0)\left(\frac{1}{1 - rz^{-1}}\right) \\
X(z) &= 5 X_o(z) \\
X_o(z) &= x(0)\left(\frac{1}{1 - r^2z^{-2}}\right) \\
x(0)\left(\frac{1}{1 - rz^{-1}}\right) &= 5 x(0)\left(\frac{1}{1 - r^2z^{-2}}\right) \\
\frac{1 - rz^{-1}}{1 - r^2z^{-2}} &= 5 \frac{1 - r^2z^{-2}}{1 - rz^{-1}} \\
(1 - r^2z^{-2}) &= 5 (1 - rz^{-1}) \\
1 - r^2z^{-2} &= 5 - 5rz^{-1} \\
4 &= 5rz^{-1} - r^2z^{-2} \\
4z^2 &= 5rz - r^2 \\
r^2 - 5rz + 4z^2 &= 0 \\
r &= \frac{5z \pm \sqrt{25z^2 - 16z^2}}{2}\\
r=4z or r=z
\end{align}
Let's assume a sequence \( x(n) \) given by \( x(n) =x(0)^n u(n) \), where \( x(0) \) is a constant and \( u(n) \) is the unit step function.

The Z-transform is given by:
\begin{align}
X(z) = \sum_{n=0}^{\infty} x(n)z^{-n} = \sum_{n=0}^{\infty} (x(0)z^{-1})^n = \frac{1}{1 - x(0)z^{-1}} 
\end{align}

This represents the Z-transform for the given sequence \( x(n) = x(0)^n u(n) \).

\section*{Desired Sum Using $X(z)$}
The Z-transform of the sequence \( x(n) = x(0)^n u(n) \) is given by:
\begin{align}
X(z) = \sum_{n=0}^{\infty} x(n)z^{-n} = \frac{1}{1 - az^{-1}} 
\end{align}

To obtain the desired sum, let's perform the inverse Z-transform by expressing \( X(z) \) in partial fractions:
\begin{align}
X(z) = \frac{1}{1 - x(0)z^{-1}} = \frac{A}{1 - az^{-1}} 
\end{align}
To find \(A\), multiply both sides by the denominator:
\begin{align}
1 = A(1 - x(0)z^{-1}) \\
A=1
\end{align}

Therefore, the partial sum using \(X(z)\) is \(x(n) = x(0)^n u(n)\).

\end{document}



