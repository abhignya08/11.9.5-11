\documentclass[journal,12pt,twocolumn]{IEEEtran}
\usepackage{cite}
\usepackage{amsmath,amssymb,amsfonts,amsthm}
\usepackage{algorithmic}
\usepackage{graphicx}
\usepackage{textcomp}
\usepackage{xcolor}
\usepackage{txfonts}
\usepackage{listings}
\usepackage{enumitem}
\usepackage{mathtools}
\usepackage{gensymb}
\usepackage{comment}
\usepackage[breaklinks=true]{hyperref}
\usepackage{tkz-euclide}
\usepackage{gvv}
\def\inputGnumericTable{}
\usepackage[latin1]{inputenc}
\usepackage{color}
\usepackage{array}
\usepackage{longtable}
\usepackage{calc}
\usepackage{multirow}
\usepackage{hhline}
\usepackage{ifthen}
\usepackage{lscape}

\newtheorem{theorem}{Theorem}[section]
\newtheorem{problem}{Problem}
\newtheorem{proposition}{Proposition}[section]
\newtheorem{lemma}{Lemma}[section]
\newtheorem{corollary}[theorem]{Corollary}
\newtheorem{example}{Example}[section]
\newtheorem{definition}[problem]{Definition}
\newcommand{\BEQA}{\begin{eqnarray}}
\newcommand{\EEQA}{\end{eqnarray}}
\newcommand{\define}{\stackrel{\triangle}{=}}
\theoremstyle{remark}
\newtheorem{rem}{Remark}
\begin{document}

\bibliographystyle{IEEEtran}
\title{Maths Assignment}
\author{Abhignya Gogula\\
        EE23BTECH11023}
\maketitle
\section*{Problem Statement}
A G.P consists of an even number of terms. If the sum of all terms is 5 times the sum of terms occupying odd places, then find its common ratio.
\section*{Solution}
\begin{table}[h!]
\centering

\begin{tabular}{|c|c|c|}
\hline
Parameter & Description & condition\\
\hline
\( n \) & Number of terms in the G.P & - \\
\hline
\( m \) & natural number & n=2m \\
\hline
\(x(0) \) & first term in the G.P & -\\
\hline
\( r \) & common ratio in the G.P & - \\
\hline
\( x(n) \) & $n+1$ th term in the G.P & $x(n)=x(0)r^{n}$\\
\hline
\( y(n) \) & sum of G.P series & $y(n)=x(0)\brak{\frac{r^{n+1}-1}{r-1}}u(n)$\\
\hline
\( w(n) \) & sum of terms in odd places & $w(n)=x(0)\brak{\frac{r^{n+1}-1}{r^2-1}}u(n)$\\
\hline
\end{tabular}




\caption{Input Parameters}
\label{11.9.5.11tab1}
\end{table}
Solving the Question in time domain:
\begin{align}
n=2m\\
x(n) &= x(0)r^{2m} \\
y(n) &= x(0)\brak{\frac{r^{2m+1}-1}{r-1}}u(2m)
\label{eq:11.9.5.11eq1}
\end{align}
\begin{align}
Y(z) &= \frac{x(0)}{\brak{1-rz^{-1}}\brak{1-z^{-1}}}
\end{align}
The sum of terms in odd places:
\begin{equation}
w(n) = x(0)\brak{\frac{r^{2m+1}-1}{r^2-1}}u(2m)
\label{eq:11.9.5.11eq2}
\end{equation}
where $r \neq \pm 1$
\begin{align}
W(z) &= \frac{x(0)}{\brak{1-rz^{\frac{-1}{2}}}\brak{1-z^{-1}}}
\end{align}
Then from \eqref{eq:11.9.5.11eq1} and \eqref{eq:11.9.5.11eq2}
\begin{align}
x(0)\brak{\frac{r^{2m+1}-1}{r-1}}u(2m) &= 5x(0)\brak{\frac{r^{2m+1}-1}{r^2-1}}u(2m)\\
r^2-5r+4 &= 0\\
r &= 4
\end{align}
\end{document}

