\documentclass{article}
\usepackage{amsmath}
\usepackage{array}

\begin{document}

\title{Maths Assignment}
\author{Abhignya Gogula\\
        EE23BTECH11023}
\maketitle

\section*{Problem Statement}
A G.P consists of an even number of terms. If the sum of all terms is 5 times the sum of terms occupying odd places, then find its common ratio.

\section*{Solution}
Let \( X(0) \) denote the first term of the geometric progression and \( r \) the common ratio. The sum of a geometric progression with \( n \) terms can be calculated using the formula:
\[ S_n = X(0) \frac{{r^n - 1}}{{r - 1}} \]

The sum of terms occupying odd places (i.e., at positions 1, 3, 5, ...) in a geometric progression can be represented as:
\[ S_{\text{odd}} = X(0) \frac{{r^{(n/2)} - 1}}{{r - 1}} \]

Given that the sum of all terms (\( S_{2n} \)) is 5 times the sum of terms occupying odd places (\( S_{\text{odd}} \)), we can set up the equation:
\[ X(0) \frac{{r^{2n} - 1}}{{r - 1}} = 5 \cdot X(0) \frac{{r^n - 1}}{{r - 1}} \]

Simplifying by canceling out the common term \( a \) and dividing both sides by \( r - 1 \):
\[ r^{2n} - 1 = 5 \cdot r^n - 5 \]
\[ r^{2n} - 5 \cdot r^n + 1 = 0 \]

Let \( x = r^n \), then the equation becomes a quadratic equation in terms of \( x \):
\[ x^2 - 5x + 1 = 0 \]

Solving this quadratic equation for \( x \), we can find \( r \) as \( x^{1/n} \). Using the quadratic formula:
\[ x = \frac{{5 \pm \sqrt{21}}}{2} \]

Since \( x = r^n \), \( r = x^{1/n} \), and considering \( n \) is a positive even number, we take the positive root:
\[ r = \left(\frac{{5 + \sqrt{21}}}{2}\right)^{1/n} \]

This gives the common ratio \( r \) in terms of \( n \), the number of terms in the geometric progression.

\section*{Input Parameters}
\begin{center}
\begin{tabular}{|c|c|}
\hline
Parameter & Description \\
\hline
\( n \) & Number of terms in the G.P (positive even integer) \\
\( X(0) \) & a \\
\hline
\end{tabular}
\end{center}

\end{document}


